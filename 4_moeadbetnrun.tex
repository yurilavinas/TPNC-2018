\section{Experiment Design}

\subsection{MOEA/D and Bet-and-Run}

In this work, the integrate of both frameworks, the MOEA/D the Bet-and-Run, is analized. First, the implementation of MOEA/D is discussed. Then the implementation of the Bet-and-Run followed by the discussion of how to integratate them.

\subsubsection{MOEA/D}

In this paper, two different MOEA/D classical variations found in the literature were studied. They are the original MOEA/D~\cite{zhang2007moea} and MOEA/D-DE~\cite{li2009multiobjective}. 

The first modification was to change the parameter control $H$ of the simplex-lattice design (SLD) that is used to generate the weight vectors W. For the 2-objective problem benchmark functions it was set as \textit{199}, while for the 3-objective problem benchmark functions, \textit{19}. Those vales for the $H$ parameter were chosen so that the number of sub-problems and the size of incumbent solutions are equal to \textit{200}, following default settings as in the works of Brockhoff et. al~\cite{brockhoff2015benchmarking} and Tanabe et. al~\cite{tanabe2018analysis}. The other modification was to use an external solution archive, that stores the best solutions found during the search process.



Here, the integration of Generic Resource Allocation - On-line Resource Allocation (GRA-ONRA), proposed in the context of MOEA/D-GRA by Zhou et. al~\cite{zhou2016all} was also considered. The resource distribution with GRA-ONRA is allocated using an adaptive strategy aiming to adjust the behavior of an algorithm in on-line manner to suit the problem in question. Although, other strategies were proposed in the work of Zhou et. al, GRA-ONRA was the one that perfomed better among all strategies proposed. This strategy distributes resources in an execution of MOEA/D given different amounts of resources to different sub-problems, following the assumption that some sub-problems can be more difficult to approximate that others.

\subsubsection{Bet-and-Run}

In this work, the Bet-and-Run framework implemented follows the structured showed by Fischetti and Monaci~\cite{fischetti2014exploiting}. One adjustment was made in this work that is to better fit Bet-and-Run to the context of MOP and MOEA/D which is defining the budget as the number of interactions, instead of using time as the budget as in the work of Fischetti and Monaci.


