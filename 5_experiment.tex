\section{Experiment Design}

The DTLZ~\cite{deb2005scalable} and the ZDT~\cite{zitzler2000comparison},test problems were used in the analysis. For the first the number of objectives used was two three. According to~\cite{deb2005scalable}, for the DTLZ problems, the number of position variables D was set to $k = 5$ for the DTLZ1 problem, $k = 7$ for the DTLZ2 problem and $k = 10$ for the other DTLZ problems, where the number of variables $D = n_f + k -1$. For the ZDT the number of variable $D = 11$.

Our limit budget was set to the maximum of iteractions, with value of 300, which leads to a number of functions evaluations of XXXXX.


The hypervolume (HV) indicator~\cite{zitzler1998multiobjective} was used. ishibuchi2018specify
Compared with their HV - normalized between 0 and 1 (based on the fair comparison paper).
30 repetitions.
box-plots 
Kruskal-Wallis (data non-normal data, used in the literature)

Configurations and Parameters

Control - Based on the common variation: MOEA/D (variation1) and MOEA/D-DE as in preset\_moead
Control and ONRA - parameters: dt = 20
Ben-and-run
Ben-and-run and ONRA - parameters: dt = 20

Dt - interval that control the resources allocation. From the proposal paper, there is no much sensibility.
Decomposition method used - SLD, with H being 199 for 2D and 19 for 3D 
number of dimensions - 60 
All other parameters are defined by  preset\_moead

Bet-and-run

Phase 1 of the bet-and-run strategy is using the epsilon indicator. 40 instances.
It needs two Pareto sets. The first is the Pareto set of a bet instance while the other is the Pareto set from the control algorithm executed with 1\% of the number of interactions. 
Phase 2 uses the 60\% rest of max interactions.

\section{Evaluation Metrics}

Evaluation Metrics

Unary Indicators

- Measure Pareto Sets independently.
- Power is restricted.
- Cannot tell in general if a set is better than another.
- Focus on problem dependent and specifics.
- Assumptions and knowledge should be specified.
1. Hyper-volume.
2. Error ratio.
3. Distance from reference set.

Binary Indicators

- Theoretically have no limitations.
- Analysis and presentation of results more difficult.

1. R1, R2, R3 indicators.
2. $\varepsilon$-Indicator.
3. Binary Hyper-volume.

Hypervolume
Considerations

- Is complete - If, and only if $HV(A) > HV(B) \implies A$ is not worse than $B$.
- Is weakly compatible - $HV(A) > HV(B) \implies \not B$ dominates $A$.
- Assumptions - All points of a Pareto Set under consideration dominate the reference point.
- @ishibuchi2018specify proposed a method to specify the reference point from a viewpoint of fair performance comparison.

Considerations
- A large population size is **always** more beneficial than a small one.
- Measures both the convergence toward the Pareto Front and the diversity of non-dominated solutions.
- A monotonic increase of the hyper-volume over time cannot always be ensured.
- For MOEA/D that is always true.

$\varepsilon$-Indicator
Considerations
- It compares 2 Pareto Sets.
- It indicates which set is better and how much better
- If A is better than B $\implies I_{\varepsilon}(B,A) > 0$.
- If $I_{\varepsilon}(A,B) \leq 0$ and $I_{\varepsilon}(B,A) > 0 \implies A$ is better than $B$.

The benchmark used are the DTLZ and the ZDT group of functions.

DTLZ are easy~\cite{bezerra2015comparing}.

