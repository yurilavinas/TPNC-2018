% This is samplepaper.tex, a sample chapter demonstrating the
% LLNCS macro package for Springer Computer Science proceedings;
% Version 2.20 of 2017/10/04
%
\documentclass[runningheads]{llncs}
%
\usepackage{graphicx}
\usepackage{amssymb}
\usepackage{amsmath}

\usepackage[ruled]{algorithm2e}
% Used for displaying a sample figure. If possible, figure files should
% be included in EPS format.
%
% If you use the hyperref package, please uncomment the following line
% to display URLs in blue roman font according to Springer's eBook style:
% \renewcommand\UrlFont{\color{blue}\rmfamily}

\begin{document}
%
\title{Bet-and-Run Strategy and MOEA/D \\ An Alternative Resource Management}
%
%\titlerunning{Abbreviated paper title}
% If the paper title is too long for the running head, you can set
% an abbreviated paper title here
%
\author{First Author\inst{1}\orcidID{0000-1111-2222-3333} \and
Second Author\inst{2,3}\orcidID{1111-2222-3333-4444} \and
Third Author\inst{3}\orcidID{2222--3333-4444-5555}}
%
\authorrunning{F. Author et al.}
% First names are abbreviated in the running head.
% If there are more than two authors, 'et al.' is used.
%
\institute{Princeton University, Princeton NJ 08544, USA \and
Springer Heidelberg, Tiergartenstr. 17, 69121 Heidelberg, Germany
\email{lncs@springer.com}\\
\url{http://www.springer.com/gp/computer-science/lncs} \and
ABC Institute, Rupert-Karls-University Heidelberg, Heidelberg, Germany\\
\email{\{abc,lncs\}@uni-heidelberg.de}}
%
\maketitle              % typeset the header of the contribution
%
\begin{abstract}
Bet-and-run strategies have been shown, both experimentally and theoretically, to be beneficial on single objective problems. Given this success and the fact that they do not take any problem knowledge into account and are not tailored to the optimization algorithms, here it is proposed an integration a bet-and-run strategy into the Multiobjective Evolutionary Algorithm based on Decomposition framework, (MOEA/D). 

MOEA/D represent a class of population-based metaheuristics for the solution of multicriteria optimizarion problems. It decomposes a multiobjective optimization problem into a set of scalar objective subproblems and solve this set in a collaborative way. 

\end{abstract}
%
%
%
%%%%%%%%%%%%%%%%%%%%%%%%%%%%%%%%%%%%%%%%%%%%%%%%%%%%%%%%%%%%%%%%%%
\section{Introduction}\label{intro}

Multiobjective Optimization Problem have $m$ multiple objective functions that must be optimized simultaneously.

Maximize$^1$ $F(x) = (f_1(x), f_2(x), ..., f_m(x))$,

subject to $x$ in $\Omega$.

- $F(x)$ objective functions;
- $f_i$ is the i-th objective to be maximized;
- $x$ is the decision vector;
- $\Omega$ is the decision space.

\footnotesize $^1$ All definitions are for maximization. Following inequalities should be reversed if the goal is to minimize.

 Many real-world scientific and engineering are MOP.
 Water quality control, Groundwater pollution re-mediation, Design of marine vehicles~\cite{coello2007evolutionary}.
 Petrol extraction.
 Hard problems: to balance the interests of the multi-objective as a whole is hard. 



Objectives may be conflicting
- The goal is to find good trade-off.

- Set of solutions.

Set of *optimum solutions* - Pareto set.

- Non-dominated solutions: no single solution provides a better trade-off in all objectives.

\begin{figure*}[!t]
\centering
\includegraphics[width=\textwidth]{images/pareto_front_diff_scalarizins_f.png}
\caption{Pareto Set - \cite{ishibuchi2009adaptation} }.
\end{figure*}


1Let $u = (u_1, ..., u_m)$ and $v = (v_1, ..., v_m)$ vectors in $\Omega$ (the decision space).
- $\forall i:u$ dominates $v$ if $f_i(u) \leq f_i(v)$ and $\exists j:f_j(u) < f_j(v)$.
- u dominates v, v is dominated by u, u is better that v.

 A point $x^*$ in $\Omega$  is called *Pareto Optimal* if no other point dominates $x^*$. 


%TODO: This needs to be changed
\begin{figure*}[!t]
\centering
```{r fig.width=3, fig.height=15,echo=FALSE}
\includegraphics[width=\textwidth]{images/pareto_dominated.png}
%\caption{Pareto Front - From: http://www.cenaero.be/Page.asp?docid=27103&langue=EN}
\end{figure*}

The set of all Pareto Optimal is called the Pareto Set. 

$P^\ast$ = $\{x \in \Omega: \nexists$ y $\in \Omega$ and $F(y) \leq F(x)\}$

 Pareto Front is the image of the Pareto Set in the objective space.
 PF = {$F(x) = (f_i(x), ..., f_m(x)): x \in P^*$}



\section{MOEA/D}\label{sec:background} 


MOEA/D represents a class of population-based meta-heuristics for solving Multi Objective Problems (MOPs).

It is based on decomposition - one kind of scalarizing function
One multi-objective problem becomes various single-objective sub-problems.
All sub-problems are solved in parallel.
A decomposition strategy generates weight vectors that defines the sub-problems.

\begin{figure*}[!t]
	\centering
	\includegraphics[width=\textwidth]{images/decomp2.png}
	\caption{Decomposition  -  2 and 3 objectives, \cite{chugh2017handling} }.
\end{figure*}

Why use decomposition?

It may be good at generating an even distribution of solutions in MOPs
It reduces the computation complexity when compared to other algorithms (NSGA-II) \cite{zhang2009performance}.
An optimal solution of a set of scalar optimization problems can be a Pareto optimal solution, under mild conditions
All solutions can be compared based on their objective function values
It is simple to find a solution to multi single-objective problems than for a multi-objective problem
Fitness assignment and diversity maintenance become easier to handle.

$f_{3}(x) = F * w_{3}$

In general, $f_{i}(x) = F * w_{i}$

\begin{figure*}[!t]
	\centering
	\includegraphics[width=\textwidth]{images/decomp.png}
	\caption{Decomposition and Aggregation Function - \cite{chugh2017handling}.}
\end{figure*}

Components of the MOEA/D

- Decomposition strategy: decomposes w/ weight vectors;
- Aggregation function:  weight vector => single-objective sub-problems;
- Neighbourhood assignment strategy: Relationship between sub-problems;
- Variation Stack: New candidates solutions;
- Update Strategy: Maintain/discard candidate solutions;
- Constraint handling method: Constraint violation;
- Termination Criteria: when to stop the search.

Variations Already Integrated

 On-line Resource Allocation - proposed in the context of MOEA/D by \cite{zhou2016all}.
 Bet-and-Run: A kind of restart strategy - in the context of single-objective problems (SOP) by \cite{friedrich2017generic}.

What is Online Resource Allocation 

- On-line Resource Allocation (ONRA) is an adaptation strategy that aim to adjust the behaviour of an algorithm in an on-line manner to suit the problem in question.


How it affects MOEA/D \cite{zhou2016all}.

- Some sub-problems can be more difficult to approximate that others. To better explore them, different computational resources are allocated to different sub-problems.

- The resources re-allocated is *the number of functions evaluations*.
- From an equal amount to every sub-problem to an amount related to the difficulty of the sub-problem.    


 Restart Strategy 


- Restart Strategy is a strategy used to avoid heavy-tailed running time distributions \cite{gomes2000heavy}.

-  If a execution of an algorithm does not conclude within a pre-determined limit or if the solution quality is unsatisfactory, we restart the algorithm @lissovoi2017theoretical.


Bet-and-Run framework 

- It is defined in @fischetti2014exploiting. as a number of short runs with randomized initial conditions, bet on the most promising run, and bring it to completion.

- To the best of our knowledge, only applied with EA in the context of SOP.

How it affects MOEA/D - \cite{lissovoi2017theoretical}.

- Initialisation can have a small beneficial effect even on very easy functions.

- Countermeasure when problems with promising and deceptive regions are encountered.

- Additional speed-up heuristic. 
%%%%%%%%%%%%%%%%%%%%%%%%%%%%%%%%%%%%%%%%%%%%%%%%%%%%%%%%%%%%%%%%%%
\section{Bet-and-Run}\label{intro}

\subsection{Restart Strategy}
Restart Strategies are a mechanism helps the algorithm to explore more in the solution area~\cite{yu2018simulated}. For instance, stochastic algorithms and randomized search heuristics may encounter some stagnation before finding a high quality solution. One way to overcome  such stagnation is to introduce a restart strategy, since it forcibly changes the search points by restoring the algorithm to its beginning~\cite{kanahara2018restart}. Also, Restart Strategy might be used to avoid heavy-tailed running time distributions \cite{gomes2000heavy}, because if a execution of an algorithm does not conclude within a pre determined limit or if the solution quality is unsatisfactory, the algorithm is restarted~\cite{lissovoi2017theoretical}. Finally, it may be considered as an additional speed-up \cite{friedrich2017generic}.

\subsection{Bet-and-Run framework}


Fischetti and Monaci~\cite{fischetti2014exploiting} investigated the Bet-and-Run framework. They defined it as a number of short runs with randomized initial conditions (the bet-phase) and then bet on the most promising run(the bet-phase) and bring it to completion. In their work, they studied the following Bet-and-Run framework:\\


\indent \textbf{Phase 1} performs \textit{k} runs of the algorithm for some short time limit \textit{$t_1$} with $t_1 \leq t/k$.\\
\indent \textbf{Phase 2} uses remaining time $t_2 = t - k*t_1$ to continue \textit{only the best run} from the first phase until time out. \\

In 2017, Lissovoi and Sudholt~\cite{lissovoi2017theoretical} analyzed this framework theoretically. They investigate it in the context of single objective problems and found that new initializations can have a small beneficial effect even on very easy functions, that this restart strategy might be an effective countermeasure when problems with promising and deceptive regions are encountered.


To the best of our knowledge, the Bet-and-Run framework was only applied with evolutionary algorithms in the context of single objective problems. 

% arrumar o where...
\section{MOEA/D and Bet-and-Run}

In this work, we propose to integrate both frameworks, the MOEA/D the Bet-and-Run.  First, the implementation of MOEA/D is discussed. Then the implementation of the Bet-and-Run followed by the discussion of how to integratate them.

\subsection{MOEA/D}

In this paper, two different MOEA/D combinations found in the literature  were studied. These combinations are the original MOEA/D~\cite{zhang2007moea} and MOEA/D-DE~\cite{li2009multiobjective}. 

The first modification was to change the parameter control $H$ of the simplex-lattice design (SLD) that is used to generate the weight vectors W. For the 2-objective problem benchmark functions it was set as \textit{199}, while for the 3-objective problem benchmark functions, \textit{19}. Those vales for the $H$ parameter were chosen so that the number of sub-problems and the size of incumbent solutions are equal to \textit{200}, following default settings as in the recent work from Tanabe et. al~\cite{tanabe2018analysis}. Na verdade eles usam varios tamanhos de populacao, e percebem que menor e melhor no inicio e pior no fim. The other modification was to use an archive, that stores all nondominated solutions found during the search process.(????????). 

We also studied the integration of On-line Resource Allocation (ONRA), proposed in the context of MOEA/D by \cite{zhou2016all}. The resource distribution when using ONRA is allocated using an adaptative strategy aiming to adjust the behaviour of an algorithm in on-line manner to suit the problem in question. Although, other strategies were proposed in the work of Zhou, ONRA was the one that perfomed better among all strategies proposed. The ONRA strategy is concerned with the distribution of resources in an execution of MOEA/D. Different amounts of resources are considered to different sub-problems, following the assumption that some sub-problems can be more difficult to approximate that others. 

\subsection{Bet-and-Run}

In this work, the Bet-and-run framework implemented follows the results found  by Friedrich et. al~\cite{friedrich2017generic}. They studied different combinations strategies that are diverse on the amount of resources assigned for phase 1 and 2. 

The best overall strategy found is the one that uses 40\% of the total budget available on short runs (phase1) and then run the most prominient one (phase 2) with the remaining 60\% of the budget found. One adjusment was made to better fit the context of MOP and MOEA/D which is defining the budget as the number of interactions, instead of using time as the budget as Friedrich et. al used.

%Phase 1 of the bet-and-run strategy is using the epsilon indicator. 40 instances.
%It needs two Pareto sets. The first is the Pareto set of a bet instance while the other is the Pareto set from the control algorithm executed with 1% of the number of interactions. 
%Phase 2 uses the 60% rest of max interactions.


%
\section{Experiment Design}

The DTLZ~\cite{deb2005scalable} and the ZDT~\cite{zitzler2000comparison},test problems were used in the analysis. For the first the number of objectives used was two three. According to~\cite{deb2005scalable}, for the DTLZ problems, the number of position variables D was set to $k = 5$ for the DTLZ1 problem, $k = 7$ for the DTLZ2 problem and $k = 10$ for the other DTLZ problems, where the number of variables $D = n_f + k -1$. For the ZDT the number of variable $D = 11$.

Our limit budget was set to the maximum of iteractions, with value of 300, which leads to a number of functions evaluations of XXXXX.


The hypervolume (HV) indicator~\cite{zitzler1998multiobjective} was used. ishibuchi2018specify
Compared with their HV - normalized between 0 and 1 (based on the fair comparison paper).
30 repetitions.
box-plots 
Kruskal-Wallis (data non-normal data, used in the literature)

Configurations and Parameters

Control - Based on the common variation: MOEA/D (variation1) and MOEA/D-DE as in preset\_moead
Control and ONRA - parameters: dt = 20
Ben-and-run
Ben-and-run and ONRA - parameters: dt = 20

Dt - interval that control the resources allocation. From the proposal paper, there is no much sensibility.
Decomposition method used - SLD, with H being 199 for 2D and 19 for 3D 
number of dimensions - 60 
All other parameters are defined by  preset\_moead

Bet-and-run

Phase 1 of the bet-and-run strategy is using the epsilon indicator. 40 instances.
It needs two Pareto sets. The first is the Pareto set of a bet instance while the other is the Pareto set from the control algorithm executed with 1\% of the number of interactions. 
Phase 2 uses the 60\% rest of max interactions.

\section{Evaluation Metrics}

Evaluation Metrics

Unary Indicators

- Measure Pareto Sets independently.
- Power is restricted.
- Cannot tell in general if a set is better than another.
- Focus on problem dependent and specifics.
- Assumptions and knowledge should be specified.
1. Hyper-volume.
2. Error ratio.
3. Distance from reference set.

Binary Indicators

- Theoretically have no limitations.
- Analysis and presentation of results more difficult.

1. R1, R2, R3 indicators.
2. $\varepsilon$-Indicator.
3. Binary Hyper-volume.

Hypervolume
Considerations

- Is complete - If, and only if $HV(A) > HV(B) \implies A$ is not worse than $B$.
- Is weakly compatible - $HV(A) > HV(B) \implies \not B$ dominates $A$.
- Assumptions - All points of a Pareto Set under consideration dominate the reference point.
- @ishibuchi2018specify proposed a method to specify the reference point from a viewpoint of fair performance comparison.

Considerations
- A large population size is **always** more beneficial than a small one.
- Measures both the convergence toward the Pareto Front and the diversity of non-dominated solutions.
- A monotonic increase of the hyper-volume over time cannot always be ensured.
- For MOEA/D that is always true.

$\varepsilon$-Indicator
Considerations
- It compares 2 Pareto Sets.
- It indicates which set is better and how much better
- If A is better than B $\implies I_{\varepsilon}(B,A) > 0$.
- If $I_{\varepsilon}(A,B) \leq 0$ and $I_{\varepsilon}(B,A) > 0 \implies A$ is better than $B$.

The benchmark used are the DTLZ and the ZDT group of functions.

DTLZ are easy~\cite{bezerra2015comparing}.


\input{6_results.tex}
\input{7_conclusion.tex}

\bibliographystyle{splncs04}
\bibliography{bib} 



\end{document}
