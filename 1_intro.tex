%%%%%%%%%%%%%%%%%%%%%%%%%%%%%%%%%%%%%%%%%%%%%%%%%%%%%%%%%%%%%%%%%%
\section{Introduction}\label{intro}

A Multiobjective Optimization Problem (MOP)  are box-constrained problems that have $m$ multiple objective functions that must be optimized simultaneously:

\begin{align}\label{min_problem}
min f(x) = (f_1(x), f_2(x), ..., f_{n_f}(x)), \text{ subject to $x$ in $\Omega$},
\end{align}

where ${n_f}$ is the number of objective functions, $x \in \mathbb{R}^{n_v}$, the decision vector, represents a candidate solution with ${n_v}$ variables, $f: \mathbb{R}^{n_v} \rightarrow \mathbb{R}^{n_v}$ is a vector of objective functions and $\Omega$ is the feasible decision space, such that:

\begin{align}
	\Omega =\{x \text{ in } \mathbb{R}^{n_v} | g_i(x) \leq 0 \text{ } \forall_i \text{ and } h_i(x) = 0 \text{ } \forall_j \},
\end{align}

Objectives often conflict with each other, therefore, no point in $\Omega$ minimizes all the objectives at the same time. Consequently, the goal of MOP solvers is to find the best trade-off that balances the different objectives in an optimal way.


Given two feasible solutions $u, v$ in $\Omega$, $u$  Pareto-dominates $v$, denoted by $f(y) \succ f(v)$, if and only if $f_k(u) \leq f_k(v), \forall_k \in \{1,..., n_f\}$ and $ f(u) \neq f(x)$. A solution $x^* \in \Omega$ is considered Pareto-Optimal if there exists no other solution $y \in \Omega$ such that $f(y) \succ f(x^*)$, i.e., if $x^*$ is non-dominated in the feasible decision space. A point is called non-dominated if no other point dominates it. That is, no single solution provides a better trade-off in all objectives.

The set of all Pareto-optimal solutions is known as the Pareto-Optimal set (PS), while the image of this set is referred to as the Pareto-optimal front (PF).\\

\begin{equation}
	PS = {x^* \in \Omega | \nexists y \in \Omega : f(y) \succ f(x^*)  },
\end{equation}

\begin{equation}
	PF = {f(x^*) | x^* \in PS }.
\end{equation}
