\section{Comparison Study}

The 7-DTLZ~\cite{deb2005scalable} and the 9-WFG~\cite{huband2006review} test problems were used to perform the comparative analysis. For both, the number of objectives used was two and three objectives. According to~\cite{deb2005scalable}, for the DTLZ problems, the number of position variables $k$ was set to $k = 5$ for the DTLZ1 problem, $k = 7$ for the DTLZ2 problem and $k = 10$ for the other DTLZ problems, where the number of variables $D = n_f + k -1$. While for the WFG problems the number of position variables D was set to $k = 2(n_f) - 1$, the number of distance variables $L$ was set to $L = 20$ and the number of dimensions is given by $D = k + L$.

The termination criteria was set to the maximum of iterations, with value of \textit{300} interactions, which leads to a number of functions evaluations of \textit{60200} in the case of the 2-objective problems, while for the 3-objective problems, the number of evaluations lead was \textit{63210}.


The hyper-volume indicator~\cite{zitzler1998multiobjective}, the epsilon indicator~\cite{zitzler2003performance}, and the Inverted Generational Distance (IGD)~\cite{zitzler2003performance} were used for evaluating the quality of the PF found by the MOEAs here studied.set of obtained non-dominated solutions $N-DS$.

The hyper-volume (HV) indicator~\cite{zitzler1998multiobjective} was used to evaluate, following the same principles and structure as in for the fine tune study.

The epsilon-indicator~\cite{zitzler2003performance} was used for evaluating the quality of two PF found. This indicator is a binary indicator that compares two Pareto Fronts and may indicate which Front is better and by how much. The first Pareto Front used is the one of the MOEAs while the second is always the \textit{control} Pareto Front given by MOEA/D or MOEA/D-DE. This second Pareto Front is called control since its only purpose is to define a fair reference front for every bet.


Two groups of combinations of algorithms were tested. The first is the MOEA/D group and consists of the following MOEAs: \textit{MOEA/D}, \textit{MOEA/D with Bet-and-Run strategy}, \textit{MOEA/D with GRA-ONRA},  \textit{MOEA/D with Bet-and-Run strategy and GRA-ONRA}.

The other group is the MOEA/D-DE group and consists of: \textit{MOEA/D-DE}, \textit{MOEA/D-DE with Bet-and-Run strategy}, \textit{MOEA/D-DE with GRA-ONRA}, \textit{MOEA/D-DE with Bet-and-Run strategy and GRA-ONRA}.

\subsubsection{GRA-ONRA}
The only extra parameter added by GRA-ONRA is the, $DT$. Here, it is set to $20$, in accord to the findings in~\cite{zhou2016all}. 


\subsection{results}

For each combinations above, the search was repeated 30 times, and the Kruskal-Wallis XXX box-plot XXXX.
%
%Configurations and Parameters
%
%Control - Based on the common variation: MOEA/D (variation1) and MOEA/D-DE as in preset\_moead
%Control and ONRA - parameters: dt = 20
%Ben-and-run
%Ben-and-run and ONRA - parameters: dt = 20
%
%Dt - interval that control the resources allocation. From the proposal paper, there is no much sensibility.
%Decomposition method used - SLD, with H being 199 for 2D and 19 for 3D 
%number of dimensions - 60 
%All other parameters are defined by  preset\_moead
%
%Bet-and-run
%
%Phase 1 of the bet-and-run strategy is using the epsilon indicator. 40 instances.
%It needs two Pareto sets. The first is the Pareto set of a bet instance while the other is the Pareto set from the control algorithm executed with 1\% of the number of interactions. 
%Phase 2 uses the 60\% rest of max interactions.
%
%\section{Evaluation Metrics}
%
%Evaluation Metrics
%
%Unary Indicators
%
%- Measure Pareto Sets independently.
%- Power is restricted.
%- Cannot tell in general if a set is better than another.
%- Focus on problem dependent and specifics.
%- Assumptions and knowledge should be specified.
%1. Hyper-volume.
%2. Error ratio.
%3. Distance from reference set.
%
%Binary Indicators
%
%- Theoretically have no limitations.
%- Analysis and presentation of results more difficult.
%
%1. R1, R2, R3 indicators.
%2. $\varepsilon$-Indicator.
%3. Binary Hyper-volume.
%
%Hypervolume
%Considerations
%
%- Is complete - If, and only if $HV(A) > HV(B) \implies A$ is not worse than $B$.
%- Is weakly compatible - $HV(A) > HV(B) \implies \not B$ dominates $A$.
%- Assumptions - All points of a Pareto Set under consideration dominate the reference point.
%- @ishibuchi2018specify proposed a method to specify the reference point from a viewpoint of fair performance comparison.
%
%Considerations
%- A large population size is **always** more beneficial than a small one.
%- Measures both the convergence toward the Pareto Front and the diversity of non-dominated solutions.
%- A monotonic increase of the hyper-volume over time cannot always be ensured.
%- For MOEA/D that is always true.
%
%$\varepsilon$-Indicator
%Considerations
%- It compares 2 Pareto Sets.
%- It indicates which set is better and how much better
%- If A is better than B $\implies I_{\varepsilon}(B,A) > 0$.
%- If $I_{\varepsilon}(A,B) \leq 0$ and $I_{\varepsilon}(B,A) > 0 \implies A$ is better than $B$.
%
%The benchmark used are the DTLZ and the ZDT group of functions.
%
%DTLZ are easy~\cite{bezerra2015comparing}.
%
