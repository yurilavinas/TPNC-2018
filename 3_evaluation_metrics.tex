\section{Evaluation Metrics}

Evaluation Metrics

Unary Indicators

- Measure Pareto Sets independently.
- Power is restricted.
- Cannot tell in general if a set is better than another.
- Focus on problem dependent and specifics.
- Assumptions and knowledge should be specified.
1. Hyper-volume.
2. Error ratio.
3. Distance from reference set.

Binary Indicators

- Theoretically have no limitations.
- Analysis and presentation of results more difficult.

1. R1, R2, R3 indicators.
2. $\varepsilon$-Indicator.
3. Binary Hyper-volume.

Hypervolume
Considerations

- Is complete - If, and only if $HV(A) > HV(B) \implies A$ is not worse than $B$.
- Is weakly compatible - $HV(A) > HV(B) \implies \not B$ dominates $A$.
- Assumptions - All points of a Pareto Set under consideration dominate the reference point.
- @ishibuchi2018specify proposed a method to specify the reference point from a viewpoint of fair performance comparison.

 Considerations
- A large population size is **always** more beneficial than a small one.
- Measures both the convergence toward the Pareto Front and the diversity of non-dominated solutions.
- A monotonic increase of the hyper-volume over time cannot always be ensured.
- For MOEA/D that is always true.

$\varepsilon$-Indicator
Considerations
- It compares 2 Pareto Sets.
- It indicates which set is better and how much better
- If A is better than B $\implies I_{\varepsilon}(B,A) > 0$.
- If $I_{\varepsilon}(A,B) \leq 0$ and $I_{\varepsilon}(B,A) > 0 \implies A$ is better than $B$.

The benchmark used are the DTLZ and the ZDT group of functions.

DTLZ are easy~\cite{bezerra2015comparing}.